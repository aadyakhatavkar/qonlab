\documentclass[aspectratio=169]{beamer}

\usepackage{amsmath,amssymb}
\usepackage{graphicx}

% Indicator function
\newcommand{\I}[1]{\mathbf{1}\{#1\}}

% Remove navigation icons
\setbeamertemplate{navigation symbols}{}

\begin{document}

%-------------------------------------------------
\begin{frame}{Simulation of Mean change}
\begin{itemize}
  \item How the \textbf{mean of a time series changes} under structural breaks?
  \item Two scenarios:
  \begin{itemize}
    \item \textbf{Single break} (one change point)
    \item \textbf{Multiple breaks} (two change points, three regimes)
  \end{itemize}
  \item Compare forecasting methods and evaluate accuracy using \textbf{RMSE}, \textbf{MAE}, and \textbf{Bias}.
\end{itemize}
\end{frame}

%-------------------------------------------------
\begin{frame}{Data Generating Process (Mean-Break)}
We simulate an SARIMA process with time-varying mean:
\[
y_t = \mu_t + \phi y_{t-1} + \varepsilon_t,
\qquad
\varepsilon_t \sim \mathcal{N}(0,\sigma^2),
\quad |\phi|<1.
\]

\medskip
\textbf{Single break:}
\[
\mu_t =
\begin{cases}
\mu_0, & t \le T_b,\\
\mu_1, & t > T_b.
\end{cases}
\]

\textbf{Multiple breaks:}
\[
\mu_t =
\begin{cases}
\mu_0, & t \le b_1,\\
\mu_1, & b_1 < t \le b_2,\\
\mu_2, & t > b_2.
\end{cases}
\]
\end{frame}

%-------------------------------------------------
\begin{frame}{Monte Carlo Forecast Design}
For each replication $i = 1,\dots,N$:
\begin{enumerate}
  \item Simulate $\{y_t^{(i)}\}_{t=1}^T$ from the DGP.
  \item Choose the forecast time after the break(s):
  \[
  t_0 =
  \begin{cases}
  T_b + g, & \text{single break},\\
  b_2 + g, & \text{multiple breaks},
  \end{cases}
  \]
  where $g$ is a fixed gap after the (last) break.
  \item Estimate each model using $\{y_1,\dots,y_{t_0-1}\}$ and compute a one-step forecast $\hat{y}_{t_0}$.
  \item Forecast error:
  \[
  e_i = y_{t_0}^{(i)} - \hat{y}_{t_0}^{(i)}.
  \]
\end{enumerate}
\end{frame}

%-------------------------------------------------
\begin{frame}{Method 1: SARIMA (Global)}
\textbf{Model class:}
\[
\text{SARIMA}(p,d,q)(P,D,Q)_s
\]
implemented in practice using SARIMAX (state-space form).

\medskip
\textbf{One-step-ahead forecast (conceptually):}
\[
\hat{y}_{t_0} = \mathbb{E}(y_{t_0} \mid y_1,\dots,y_{t_0-1}; \hat{\theta})
\]
where $\hat{\theta}$ are parameters estimated on the full training sample.

\medskip
\textbf{Key point:} structural breaks in $\mu_t$ are not explicitly modelled.
\end{frame}

%-------------------------------------------------
\begin{frame}{Method 2: Rolling SARIMA}
\textbf{Rolling estimation:} re-estimate the model using only the most recent window of size $w$:
\[
\mathcal{W}_{t_0-1} = \{y_{t_0-w}, \dots, y_{t_0-1}\}.
\]

\medskip
\textbf{One-step forecast:}
\[
\hat{y}^{\text{roll}}_{t_0} = \mathbb{E}(y_{t_0} \mid y_{t_0-w},\dots,y_{t_0-1}).
\]

\medskip
\textbf{Motivation:}
\begin{itemize}
  \item adapts faster after breaks by down-weighting older regimes;
  \item trade-off: fewer observations may increase estimation noise.
\end{itemize}
\end{frame}

%-------------------------------------------------
\begin{frame}{Method 3: SARIMA + Break Dummies (Oracle)}
\textbf{Idea:} include break dummies as exogenous regressors (break dates assumed known).

\medskip
\textbf{Single break dummy:}
\[
d_t = \I{t>T_b},
\qquad
y_t = c + \phi y_{t-1} + \delta d_t + u_t.
\]

\medskip
\textbf{Multiple breaks dummies:}
\[
d_{1,t}=\I{t>b_1}, \qquad d_{2,t}=\I{t>b_2},
\]
\[
y_t = c + \phi y_{t-1} + \delta_1 d_{1,t} + \delta_2 d_{2,t} + u_t.
\]

\medskip
\textbf{Key point:} dummy variables allow the model intercept (mean) to shift at the break date(s).
\end{frame}

%-------------------------------------------------
\begin{frame}{Estimated Breaks and Evaluation Metrics}
\textbf{Estimated break(s) (grid search idea):}
\begin{itemize}
  \item Single break: choose $\hat{T}_b$ minimizing pre- and post-break fit error.
  \[
  \hat{T}_b = \arg\min_{T_b}\Big(\mathrm{SSE}_{\text{pre}}(T_b)+\mathrm{SSE}_{\text{post}}(T_b)\Big).
  \]
  \item Two breaks: choose $(\hat{b}_1,\hat{b}_2)$ minimizing total three-segment fit error.
\end{itemize}

\medskip
\textbf{Evaluation metrics (from errors $\{e_i\}_{i=1}^N$):}
\[
\text{Bias}=\frac{1}{N}\sum_{i=1}^N e_i,\quad
\text{MAE}=\frac{1}{N}\sum_{i=1}^N |e_i|,\quad
\text{RMSE}=\sqrt{\frac{1}{N}\sum_{i=1}^N e_i^2}.
\]

\medskip
\textbf{Ranking rule:} lower RMSE indicates better forecast accuracy.
\end{frame}
% =========================================================
% SINGLE-BREAK (SHORT + BIG GRAPHICS VERSION)
% Order:
% 1) RMSE + MAE
% 2) Bias + Seasonal series
% 3) Numerical table (end)
% =========================================================

%-------------------------------------------------
\begin{frame}{Single-break: Forecast Accuracy (RMSE and MAE)}
\centering
\includegraphics[width=0.48\paperwidth]{single_break_RMSE.png}
\hspace{0.4cm}
\includegraphics[width=0.48\paperwidth]{single_break_MAE.png}

\vspace{0.25cm}
\Large Lower values indicate better forecast accuracy.
\end{frame}

%-------------------------------------------------
\begin{frame}{Single-break: Bias and Example Simulated Series}
\centering
\includegraphics[width=0.48\paperwidth]{single_break_Bias.png}
\hspace{0.4cm}
\includegraphics[width=0.48\paperwidth]{single_break_seasonal_series.png}

\vspace{0.25cm}
\Large Bias closer to $0$ is preferred. Dashed line shows $T_b=150$.
\end{frame}

%-------------------------------------------------
\begin{frame}{Single-break: Numerical Results (Monte Carlo, $N=200$)}

\centering
\small
\begin{tabular}{lccccc}
\hline
\textbf{Method} & \textbf{RMSE} & \textbf{MAE} & \textbf{Bias} & \textbf{N} & \textbf{Fails} \\
\hline
SARIMA + Break Dummy (oracle $T_b$)
& 1.455137 & 1.194346 & 1.005753 & 200 & 0 \\

Simple Exp.\ Smoothing (SES)
& 1.496262 & 1.224923 & 1.015489 & 200 & 0 \\

SARIMA Rolling
& 1.525482 & 1.257133 & 1.028853 & 200 & 0 \\

SARIMA + Estimated Break (grid)
& 1.634600 & 1.367799 & 1.243084 & 200 & 0 \\

SARIMA Global
& 1.692063 & 1.423220 & 1.301679 & 200 & 0 \\
\hline
\end{tabular}

\vspace{0.35cm}

\textbf{Conclusion (RMSE criterion):}  
Best method is \textbf{SARIMA + Break Dummy (oracle $T_b$)}.

\end{frame}
%------------------------------------------------------------
\begin{frame}{Comparison Table (Single vs Multiple Breaks, $N=200$)}
\large
\centering

\begin{tabular}{lccccc l}
\hline
\textbf{Method} & \textbf{RMSE} & \textbf{MAE} & \textbf{Bias} & \textbf{N} & \textbf{Fails} & \textbf{Scenario} \\
\hline
SARIMA + 2 Break Dummies (oracle)        & 0.985466 & 0.781121 &  0.105836 & 200 & 0 & Multiple breaks \\
SARIMA Global                            & 1.041633 & 0.835788 & -0.044973 & 200 & 0 & Multiple breaks \\
SARIMA + 2 Breaks Estimated (grid)       & 1.046196 & 0.844797 & -0.125415 & 200 & 0 & Multiple breaks \\
Holt--Winters Seasonal Smoothing         & 1.093613 & 0.857217 & -0.001187 & 200 & 0 & Multiple breaks \\
SARIMA Rolling                           & 1.121625 & 0.884177 &  0.298843 & 200 & 0 & Multiple breaks \\
\hline
SARIMA + Break Dummy (oracle $T_b$)      & 1.455137 & 1.194346 &  1.005753 & 200 & 0 & Single break \\
SES (level smoothing)                    & 1.496262 & 1.224923 &  1.015489 & 200 & 0 & Single break \\
SARIMA Rolling                           & 1.525482 & 1.257133 &  1.028853 & 200 & 0 & Single break \\
SARIMA + Estimated Break (grid)          & 1.634600 & 1.367799 &  1.243084 & 200 & 0 & Single break \\
SARIMA Global                            & 1.692063 & 1.423220 &  1.301679 & 200 & 0 & Single break \\
\hline
\end{tabular}

\vspace{0.4cm}

\textbf{Best method for Single break:} SARIMA + Break Dummy (oracle $T_b$) (RMSE=1.4551, MAE=1.1943)

\vspace{0.15cm}

\textbf{Best method for Multiple breaks:} SARIMA + 2 Break Dummies (oracle) (RMSE=0.9855, MAE=0.7811)

\end{frame}
%========================================================
% Comparison: Single vs Multiple Breaks (Visual Results)
%========================================================

\begin{frame}{RMSE Comparison: Single vs Multiple Breaks}
\centering
\hspace*{-0.5cm}
\includegraphics[width=0.90\paperwidth]{compare_RMSE_single_vs_multiple.png}
\end{frame}

%--------------------------------------------------------

\begin{frame}{MAE Comparison: Single vs Multiple Breaks}
\centering
\hspace*{-0.5cm}
\includegraphics[width=0.90\paperwidth]{compare_MAE_single_vs_multiple.png}
\end{frame}

%--------------------------------------------------------

\begin{frame}{Example Series: Multiple Breaks with Seasonality}
\centering
\hspace*{-0.5cm}
\includegraphics[width=0.90\paperwidth]{multiple_break_seasonal_series.png}

\vspace{0.3cm}
\Large Dashed vertical lines indicate break points
$b_1 = 100$ and $b_2 = 200$.
\end{frame}



\end{document}
