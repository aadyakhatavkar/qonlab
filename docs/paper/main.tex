%==============================================================================
% STRUCTURAL BREAK FORECASTING: A MONTE CARLO STUDY
% Research Module in Econometrics and Statistics
% University of Bonn
%==============================================================================

\documentclass[12pt,a4paper]{article}

%------------------------------------------------------------------------------
% PACKAGES
%------------------------------------------------------------------------------
\usepackage[utf8]{inputenc}
\usepackage[T1]{fontenc}
\usepackage{amsmath,amssymb,amsthm}
\usepackage{mathtools}
\usepackage{graphicx}
\usepackage{booktabs}
\usepackage{hyperref}
\usepackage{cleveref}
\usepackage[margin=1in]{geometry}
\usepackage{natbib}
\usepackage{setspace}
\usepackage{float}
\usepackage{subcaption}
\usepackage{xcolor}
\usepackage{enumitem}

% Line spacing
\onehalfspacing

%------------------------------------------------------------------------------
% CUSTOM COMMANDS
%------------------------------------------------------------------------------
\newcommand{\E}{\mathbb{E}}
\newcommand{\Var}{\text{Var}}
\newcommand{\iid}{\stackrel{\text{iid}}{\sim}}
\newcommand{\ind}{\mathbf{1}}
\DeclareMathOperator*{\argmin}{arg\,min}

%------------------------------------------------------------------------------
% TITLE AND AUTHORS
%------------------------------------------------------------------------------
\title{%
    \textbf{Structural Break Forecasting:\\A Monte Carlo Study}\\[1em]
    \large Research Module in Econometrics and Statistics\\
    Fundamentals of Monte Carlo Simulations in Data Science
}
\author{%
    Aadya Khatavkar\\
    University of Bonn\\
    \texttt{s38akhat@uni-bonn.de}
}
\date{Winter Semester 2025/26}

%==============================================================================
\begin{document}
%==============================================================================

\maketitle

\begin{abstract}
This study evaluates forecasting performance under \textbf{structural breaks} using Monte Carlo simulations. We implement data-generating processes for three break types: variance breaks (volatility shifts), mean breaks (intercept shifts), and parameter breaks (AR coefficient shifts). Forecasting methods include global ARIMA, rolling-window ARIMA with automatic order selection, GARCH, post-break estimation, and Markov switching models. We extend the analysis to heavy-tailed (Student-$t$) distributions and implement optimal window selection via grid search following Pesaran (2013). Evaluation uses both point forecast metrics (RMSE, MAE, Bias) and uncertainty quantification (Coverage, Log-score). Results inform practical guidance on method selection under parameter instability.

\vspace{1em}
\noindent\textbf{Keywords:} Structural breaks, Monte Carlo simulation, ARIMA, GARCH, rolling window, variance breaks, heavy tails
\end{abstract}

\newpage
\tableofcontents
\newpage

%==============================================================================
\section{Introduction}
\label{sec:introduction}
%==============================================================================

Time series forecasting faces a fundamental challenge when the data-generating process undergoes \emph{structural breaks}---discrete parameter changes at specific points in time. Such breaks are pervasive in economic and financial data, arising from policy changes, financial crises, and regime shifts.

This project develops a comprehensive Monte Carlo framework to evaluate forecasting methods under three types of structural breaks:
\begin{enumerate}
    \item \textbf{Variance breaks:} Shifts in innovation volatility ($\sigma_1^2 \to \sigma_2^2$)
    \item \textbf{Mean breaks:} Shifts in the intercept ($\mu_0 \to \mu_1$)
    \item \textbf{Parameter breaks:} Shifts in the AR coefficient ($\phi_1 \to \phi_2$)
\end{enumerate}

The research questions are:
\begin{enumerate}
    \item How do different forecasting methods perform under each break type?
    \item What is the optimal rolling window size for different break magnitudes?
    \item How do heavy-tailed distributions affect forecasting accuracy?
    \item Can adaptive methods match oracle specifications that know break dates?
\end{enumerate}

%==============================================================================
\section{Literature Review}
\label{sec:literature}
%==============================================================================

The challenge of forecasting under structural instability has been a central theme in econometrics for decades. \citet{stock1996} provide extensive empirical evidence that structural instability is pervasive in macroeconomic time series, showing that roughly half of the series they examined exhibited significant breaks. This instability often leads to what \citet{clements1998} term ``forecast breakdown,'' where the out-of-sample performance of a model deteriorates significantly relative to its in-sample fit. As noted in the seminal work of \citet{perron1989}, ignoring these breaks can lead to spurious results and a fundamental misunderstanding of data persistence. Furthermore, \citet{wang2013} demonstrate that structural breaks can often ``mimic'' long memory, leading to a spurious long-memory effect where shifts in the mean are incorrectly interpreted as permanent stochastic memory.

Theoretical foundations for detecting and estimating structural change were notably advanced by \citet{bai1998, bai2003} and \citet{andrews1993}, who developed tests for multiple structural changes with unknown break points using global minimization of the sum of squared residuals. These structural breaks typically manifest in three forms: (i) sudden level shifts in the mean; (ii) dynamic persistence breaks; and (iii) volatility breaks. In the context of forecasting, \citet{pesaran2013} and \citet{rossi2013} have provided comprehensive frameworks for evaluating predictive accuracy when the underlying data-generating process is unstable. \citet{pesaran2013} specifically addresses the trade-off between bias and variance when choosing window sizes for rolling estimators. However, applying these tests in small samples remains a challenge due to size distortions. \citet{antoshin2008} suggest an adaptive approach using Monte Carlo simulations to calculate sample-specific critical values ($N=50$), improving detection accuracy in constrained environments.

Alternative approaches to handling breaks include Markov-switching models, popularized by \citet{hamilton1989}, which model regime changes as transitions between latent states. For variance breaks, the GARCH framework of \citet{engle1982} and \citet{bollerslev1986} remains the standard for modeling time-varying volatility, although its performance relative to simple rolling indicators after discrete structural shifts remains an area of active study.

Our study contributes to this literature by providing a systematic Monte Carlo comparison of these methods across three distinct break types---mean, variance, and parameter shifts---while also accounting for heavy-tailed innovation distributions and seasonal patterns through SARIMA extensions.

%==============================================================================
\section{Data-Generating Processes}
\label{sec:dgp}
%==============================================================================

All DGPs are based on AR(1) processes with structural breaks at known points.

%------------------------------------------------------------------------------
\subsection{Variance Break DGP}
\label{subsec:dgp_variance}
%------------------------------------------------------------------------------

The variance-break DGP is:
\begin{equation}
    y_t = \mu + \phi y_{t-1} + \varepsilon_t, \quad \varepsilon_t \sim N(0, \sigma_t^2),
    \label{eq:dgp_variance}
\end{equation}
where:
\begin{equation}
    \sigma_t^2 = 
    \begin{cases}
        \sigma_1^2 & \text{if } t \leq T_b, \\
        \sigma_2^2 & \text{if } t > T_b.
    \end{cases}
\end{equation}

\textbf{Parameters:} $T = 400$, $T_b = 200$, $\phi = 0.6$, $\sigma_1 = 1.0$, $\sigma_2/\sigma_1 \in \{1.5, 2.0, 3.0, 5.0\}$.

%------------------------------------------------------------------------------
\subsection{Mean Break DGP}
\label{subsec:dgp_mean}
%------------------------------------------------------------------------------

The mean-break DGP is:
\begin{equation}
    y_t = \mu_t + \phi y_{t-1} + \varepsilon_t, \quad \varepsilon_t \iid N(0, \sigma^2),
    \label{eq:dgp_mean}
\end{equation}
where:
\begin{equation}
    \mu_t = 
    \begin{cases}
        \mu_0 & \text{if } t \leq T_b, \\
        \mu_1 & \text{if } t > T_b.
    \end{cases}
\end{equation}

\textbf{Parameters:} $T = 300$, $T_b = 150$, $\phi = 0.6$, $\mu_0 = 0$, $\mu_1 = 2$.

%------------------------------------------------------------------------------
\subsection{Parameter Break DGP}
\label{subsec:dgp_parameter}
%------------------------------------------------------------------------------

The parameter-break DGP is:
\begin{equation}
    y_t = \phi_t y_{t-1} + \varepsilon_t, \quad \varepsilon_t \iid N(0, \sigma^2),
    \label{eq:dgp_parameter}
\end{equation}
where:
\begin{equation}
    \phi_t = 
    \begin{cases}
        \phi_1 & \text{if } t \leq T_b, \\
        \phi_2 & \text{if } t > T_b.
    \end{cases}
\end{equation}

\textbf{Parameters:} $T = 400$, $T_b = 200$, $\phi_1 = 0.2$, $\phi_2 = 0.9$.

%------------------------------------------------------------------------------
\subsection{Heavy-Tailed Innovations}
\label{subsec:heavy_tails}
%------------------------------------------------------------------------------

For robustness, we consider Student-$t$ distributed innovations:
\begin{equation}
    \varepsilon_t = \frac{z_t}{\sqrt{\nu/(\nu-2)}}, \quad z_t \sim t_\nu.
\end{equation}

The denominator $\sqrt{\nu/(\nu-2)}$ standardizes to unit variance, enabling fair comparison with Gaussian innovations. Default: $\nu = 3$ (heavy tails).

%==============================================================================
\section{Forecasting Methods}
\label{sec:methods}
%==============================================================================

We implement six forecasting approaches.

%------------------------------------------------------------------------------
\subsection{Global ARIMA}
\label{subsec:global_arima}
%------------------------------------------------------------------------------

Fits ARIMA$(p,d,q)$ on \textbf{all training data}. Order is either fixed or auto-selected via AIC:
\begin{equation}
    (p^*, d^*, q^*) = \argmin_{p,d,q} \text{AIC}(p,d,q).
\end{equation}

\textbf{Limitation:} After breaks, estimates are contaminated by pre-break data.

%------------------------------------------------------------------------------
\subsection{Rolling-Window ARIMA}
\label{subsec:rolling_arima}
%------------------------------------------------------------------------------

Estimates ARIMA using only the most recent $w$ observations:
\begin{equation}
    \text{Training: } \{y_{t-w+1}, \ldots, y_t\}.
\end{equation}

\textbf{Key trade-off (Pesaran 2013):}
\begin{itemize}[nosep]
    \item Small $w$: Fast adaptation, high variance
    \item Large $w$: Low variance, includes pre-break data
\end{itemize}

Optimal $w$ depends on break magnitude---larger breaks favor smaller windows.

%------------------------------------------------------------------------------
\subsection{GARCH(1,1)}
\label{subsec:garch}
%------------------------------------------------------------------------------

Models conditional variance directly:
\begin{align}
    y_t &= \mu + \varepsilon_t, \quad \varepsilon_t = \sigma_t z_t, \\
    \sigma_t^2 &= \omega + \alpha \varepsilon_{t-1}^2 + \beta \sigma_{t-1}^2.
\end{align}

\textbf{Strength:} Naturally adapts to volatility changes via $\alpha$ and $\beta$.

%------------------------------------------------------------------------------
\subsection{Post-Break ARIMA}
\label{subsec:postbreak}
%------------------------------------------------------------------------------

\begin{enumerate}[nosep]
    \item Estimate break point $\hat{T}_b$ via SSE minimization
    \item Fit ARIMA on post-break data: $\{y_{\hat{T}_b+1}, \ldots, y_t\}$
\end{enumerate}

Falls back to global if insufficient post-break observations.

%------------------------------------------------------------------------------
\subsection{Averaged Window}
\label{subsec:averaged}
%------------------------------------------------------------------------------

Averages forecasts across multiple window sizes to reduce sensitivity to window choice:
\begin{equation}
    \hat{y}_{t+h} = \frac{1}{K} \sum_{k=1}^{K} \hat{y}_{t+h}^{(w_k)}.
\end{equation}

%------------------------------------------------------------------------------
\subsection{Markov Switching}
\label{subsec:markov}
%------------------------------------------------------------------------------

Regime-switching model with unobserved state $s_t \in \{1, 2\}$:
\begin{equation}
    y_t = c_{s_t} + \phi y_{t-1} + \varepsilon_t, \quad P(s_t = j | s_{t-1} = i) = p_{ij}.
\end{equation}

\textbf{Caveat:} Numerically sensitive; convergence failures common in MC loops.

%==============================================================================
\section{Monte Carlo Design}
\label{sec:montecarlo}
%==============================================================================

%------------------------------------------------------------------------------
\subsection{Simulation Procedure}
\label{subsec:mc_procedure}
%------------------------------------------------------------------------------

For each scenario:
\begin{enumerate}[nosep]
    \item Generate $\{y_t\}_{t=1}^{T}$ from relevant DGP
    \item Split: training $\{y_1, \ldots, y_{T-h}\}$, test $\{y_{T-h+1}, \ldots, y_T\}$
    \item Fit each method on training data
    \item Compute $h$-step forecasts and intervals
    \item Evaluate metrics
    \item Repeat for $N = 200$ replications
\end{enumerate}

%------------------------------------------------------------------------------
\subsection{Grid Search for Optimal Window}
\label{subsec:grid_search}
%------------------------------------------------------------------------------

Following Pesaran (2013), we search over:
\begin{itemize}[nosep]
    \item Window sizes: $w \in \{20, 50, 100, 150, 200\}$
    \item Break magnitudes: $\sigma_2/\sigma_1 \in \{1.5, 2.0, 3.0, 5.0\}$
\end{itemize}

This produces a \textbf{loss surface} for optimal window selection.

\textbf{Practitioner note:} Grid search informs fixed-window policies but should not be applied adaptively in real-time (look-ahead bias).

%==============================================================================
\section{Evaluation Metrics}
\label{sec:metrics}
%==============================================================================

%------------------------------------------------------------------------------
\subsection{Point Forecast Metrics}
\label{subsec:point_metrics}
%------------------------------------------------------------------------------

Let $e_i = y_i - \hat{y}_i$ denote forecast errors.

\textbf{RMSE:}
\begin{equation}
    \text{RMSE} = \sqrt{\frac{1}{N} \sum_{i=1}^{N} e_i^2}
\end{equation}

\textbf{MAE:}
\begin{equation}
    \text{MAE} = \frac{1}{N} \sum_{i=1}^{N} |e_i|
\end{equation}

\textbf{Bias:}
\begin{equation}
    \text{Bias} = \frac{1}{N} \sum_{i=1}^{N} e_i
\end{equation}

%------------------------------------------------------------------------------
\subsection{Uncertainty Metrics}
\label{subsec:unc_metrics}
%------------------------------------------------------------------------------

\textbf{Interval Coverage:}
\begin{equation}
    \text{Coverage}_\alpha = \frac{1}{N} \sum_{i=1}^{N} \ind\left( y_i \in \text{CI}_\alpha \right)
\end{equation}

\textbf{Log-Score (proper scoring rule):}
\begin{equation}
    \text{LogScore} = \frac{1}{N} \sum_{i=1}^{N} \left[ -\frac{1}{2}\log(2\pi\hat{\sigma}_i^2) - \frac{e_i^2}{2\hat{\sigma}_i^2} \right]
\end{equation}

%==============================================================================
\section{Implementation Summary}
\label{sec:implementation}
%==============================================================================

%------------------------------------------------------------------------------
\subsection{Code Organization}
\label{subsec:code}
%------------------------------------------------------------------------------

\begin{table}[H]
\centering
\caption{Module Overview}
\begin{tabular}{lll}
\toprule
Module & Location & Key Functions \\
\midrule
DGPs & \texttt{dgps/static.py} & \texttt{simulate\_variance\_break()}, etc. \\
Forecasters & \texttt{estimators/forecasters.py} & ARIMA, GARCH, Markov \\
MC Engine & \texttt{analyses/simulations.py} & \texttt{mc\_variance\_breaks()} \\
Visualization & \texttt{analyses/plots.py} & Loss surfaces, comparisons \\
\bottomrule
\end{tabular}
\end{table}

%------------------------------------------------------------------------------
\subsection{Key Technical Features}
\label{subsec:features}
%------------------------------------------------------------------------------

\begin{enumerate}
    \item \textbf{Automatic ARIMA order selection} via AIC/BIC grid search
    \item \textbf{Heavy-tailed distributions} with standardized Student-$t$ innovations
    \item \textbf{Unified simulation engine} handling all break types
    \item \textbf{Realized volatility functions} for empirical applications
    \item \textbf{Scenario-based configuration} via JSON files
\end{enumerate}

%==============================================================================
\section{Results}
\label{sec:results}
%==============================================================================

% Placeholder tables - to be filled with simulation results

\begin{table}[H]
\centering
\caption{Variance Break: Method Comparison (Placeholder)}
\label{tab:variance_results}
\begin{tabular}{lcccccc}
\toprule
Method & RMSE & MAE & Bias & Cov80 & Cov95 & LogScore \\
\midrule
ARIMA Global & --- & --- & --- & --- & --- & --- \\
ARIMA Rolling & --- & --- & --- & --- & --- & --- \\
GARCH(1,1) & --- & --- & --- & --- & --- & --- \\
ARIMA Post-Break & --- & --- & --- & --- & --- & --- \\
\bottomrule
\end{tabular}
\end{table}

\begin{table}[H]
\centering
\caption{Loss Surface: RMSE by Window and Break Magnitude (Placeholder)}
\label{tab:loss_surface}
\begin{tabular}{lcccc}
\toprule
Window & $\sigma_2 = 1.5\sigma_1$ & $\sigma_2 = 2\sigma_1$ & $\sigma_2 = 3\sigma_1$ & $\sigma_2 = 5\sigma_1$ \\
\midrule
$w = 20$ & --- & --- & --- & --- \\
$w = 50$ & --- & --- & --- & --- \\
$w = 100$ & --- & --- & --- & --- \\
$w = 200$ & --- & --- & --- & --- \\
\bottomrule
\end{tabular}
\end{table}

%==============================================================================
\section{Conclusion}
\label{sec:conclusion}
%==============================================================================

This project provides a comprehensive Monte Carlo framework for evaluating forecasting under structural breaks. Key contributions:

\begin{enumerate}
    \item \textbf{Unified DGP framework} for variance, mean, and parameter breaks
    \item \textbf{Multiple forecasting methods} from simple ARIMA to Markov switching
    \item \textbf{Heavy-tailed extensions} for realistic financial data
    \item \textbf{Optimal window selection} via Pesaran (2013) grid search
    \item \textbf{Both point and probabilistic} evaluation metrics
\end{enumerate}

\textbf{Practical implications:}
\begin{itemize}
    \item GARCH adapts well to variance breaks
    \item Rolling windows outperform global estimation under breaks
    \item Optimal window decreases with break magnitude
    \item Heavy tails require larger samples for stable estimation
\end{itemize}

%------------------------------------------------------------------------------
\subsection{Future Work}
\label{subsec:future}
%------------------------------------------------------------------------------

\begin{itemize}
    \item S\&P 500 realized volatility application (Thomson Reuters Eikon)
    \item Multi-step ahead forecasting
    \item ARIMA + GARCH ensemble methods
    \item Online break detection and adaptive windowing
\end{itemize}

%==============================================================================
% REFERENCES
%==============================================================================
\newpage
\bibliographystyle{apalike}

\begin{thebibliography}{99}

\bibitem[Andrews(1993)]{andrews1993}
Andrews, D. W. K. (1993).
\newblock Tests for parameter instability and structural change with unknown change point.
\newblock \emph{Econometrica}, 61(4):821--856.

\bibitem[Bai and Perron(1998)]{bai1998}
Bai, J. and Perron, P. (1998).
\newblock Estimating and testing linear models with multiple structural changes.
\newblock \emph{Econometrica}, 66(1):47--78.

\bibitem[Bollerslev(1986)]{bollerslev1986}
Bollerslev, T. (1986).
\newblock Generalized autoregressive conditional heteroskedasticity.
\newblock \emph{Journal of Econometrics}, 31(3):307--327.

\bibitem[Box and Jenkins(1970)]{box1970}
Box, G.~E. and Jenkins, G.~M. (1970).
\newblock \emph{Time Series Analysis: Forecasting and Control}.
\newblock Holden-Day.

\bibitem[Clark and McCracken(2011)]{clark2011}
Clark, T. E. and McCracken, M. W. (2011).
\newblock Averaging forecasts from VARs with uncertain instabilities.
\newblock \emph{Journal of Applied Econometrics}, 26(3):493--514.

\bibitem[Clements and Hendry(1998)]{clements1998}
Clements, M. P. and Hendry, D. F. (1998).
\newblock \emph{Forecasting Economic Time Series}.
\newblock Cambridge University Press.

\bibitem[Engle(1982)]{engle1982}
Engle, R.~F. (1982).
\newblock Autoregressive conditional heteroscedasticity with estimates of the variance of United Kingdom inflation.
\newblock \emph{Econometrica}, 50(4):987--1007.

\bibitem[Francq and Zakoïan(2019)]{francq2019}
Francq, C. and Zakoïan, J.~M. (2019).
\newblock \emph{GARCH Models}.
\newblock Wiley, 2nd edition.

\bibitem[Hamilton(1989)]{hamilton1989}
Hamilton, J.~D. (1989).
\newblock A new approach to the economic analysis of nonstationary time series.
\newblock \emph{Econometrica}, 57(2):357--384.

\bibitem[Pesaran(2013)]{pesaran2013}
Pesaran, M.~H. (2013).
\newblock The role of structural breaks in forecasting.
\newblock In \emph{Handbook of Economic Forecasting}, volume 2B, pages 1159--1191. Elsevier.

\bibitem[Rossi(2013)]{rossi2013}
Rossi, B. (2013).
\newblock Advances in forecasting under instability.
\newblock In \emph{Handbook of Economic Forecasting}, volume 2B, pages 1203--1324. Elsevier.

\bibitem[Stock and Watson(1996)]{stock1996}
Stock, J. H. and Watson, M. W. (1996).
\newblock Evidence on structural instability in macroeconomic time series relations.
\newblock \emph{Journal of Business \& Economic Statistics}, 14(1):11--30.

\end{thebibliography}

%==============================================================================
\end{document}
%==============================================================================
