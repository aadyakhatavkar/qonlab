\subsection{Single Parameter Break}

We begin with the deterministic break in persistence, where the autoregressive coefficient shifts from $\phi = 0.2$ to $\phi = 0.9$ at $T_b$. This change represents a substantial alteration in dynamic behavior, moving from weak serial dependence to near-unit-root persistence. Tables~5--7 report results under Gaussian and heavy-tailed innovations. The discussion proceeds in terms of the mean forecast performance (RMSE and MAE), bias, and forecast error variance.

Under Gaussian errors, the Markov-switching AR (MS-AR) model achieves the lowest RMSE (1.0735), followed by Rolling SARIMA (1.0950), while Global SARIMA performs worse (1.1702). The same ranking holds for MAE, indicating that allowing for regime-dependent persistence improves overall forecast accuracy. Bias is small across all models, suggesting that the gains are not driven by systematic correction of forecast direction. Instead, improvements arise primarily from reductions in dispersion. Forecast error variance declines from 1.3685 under Global SARIMA to 1.1512 under MS-AR, confirming that explicit regime modeling enhances precision rather than eliminating bias.

When innovations follow a Student-$t$ distribution with three degrees of freedom, the ranking changes. Rolling SARIMA achieves the lowest RMSE (0.9106), outperforming MS-AR (1.0502) and Global SARIMA (1.0931). The reduction in forecast error variance is particularly pronounced for the rolling specification (0.8287 compared to 1.1027 for MS-AR). Bias remains small in magnitude for all models. Under strongly heavy-tailed shocks, rolling estimation appears more robust, likely because it adapts mechanically to recent observations without relying on likelihood-based regime classification, which may be sensitive to extreme realizations.

For Student-$t$ innovations with five degrees of freedom, the results are intermediate. MS-AR again delivers the lowest RMSE (0.9653), though the margin relative to Rolling SARIMA (0.9781) is modest. Forecast error variances are closer across models than in the Gaussian case, and bias remains small and stable. Compared to the $t(3)$ case, the deterioration in MS-AR performance is less pronounced, indicating that moderate deviations from normality do not substantially weaken the benefits of explicit regime modeling.

Across all innovation distributions, Global SARIMA consistently exhibits the highest RMSE and forecast error variance. Differences in bias remain limited throughout. The principal effect of structural adaptation is therefore variance reduction rather than systematic correction of mean forecasts. Overall, the single-break results indicate that modeling regime-dependent persistence improves forecast stability under Gaussian and moderately heavy-tailed shocks, while rolling estimation provides greater robustness under strongly heavy-tailed disturbances.

\subsection{Recurring Parameter Breaks}

We next consider stochastic regime switching, where the autoregressive coefficient alternates between $\phi_0 = 0.2$ and $\phi_1 = 0.9$ according to a two-state Markov process. Tables~8--10 report results for persistence levels $p = 0.90, 0.95, 0.99$, corresponding to increasing expected regime durations. As before, we evaluate mean forecast performance, bias, and forecast error variance.

\begin{figure}[htbp]
\centering
\includegraphics[width=0.9\textwidth]{../../outputs/figures/dgp/recurring_parameter_dgp_persistence_combined.png}
\caption{Recurring parameter-break DGP across persistence levels $p \in \{0.90,0.95,0.99\}$, combined in one figure.}
\label{fig:dgp_recurring_parameter_persistence}
\end{figure}

\begin{table}[htbp]
\centering
\caption{Recurring Parameter Breaks: combined results by persistence ($N=300$ each)}
\label{tab:recurring_parameter_combined}
\begin{tabular}{lccccc}
\hline
Persistence & Method & RMSE & MAE & Bias & Var(error) \\
\hline
0.90 & MS AR & 1.1426 & 0.8922 & 0.0253 & 1.3049 \\
0.90 & Global SARIMA & 1.1695 & 0.9117 & 0.0041 & 1.3676 \\
0.90 & Rolling SARIMA & 1.1875 & 0.9257 & 0.0059 & 1.4100 \\
\hline
0.95 & MS AR & 1.0778 & 0.8570 & 0.0408 & 1.1600 \\
0.95 & Global SARIMA & 1.1238 & 0.8952 & -0.0114 & 1.2627 \\
0.95 & Rolling SARIMA & 1.1215 & 0.8990 & -0.0027 & 1.2578 \\
\hline
0.99 & MS AR & 1.0318 & 0.8043 & -0.0102 & 1.0645 \\
0.99 & Global SARIMA & 1.0974 & 0.8515 & -0.0582 & 1.2009 \\
0.99 & Rolling SARIMA & 1.0668 & 0.8408 & -0.0122 & 1.1380 \\
\hline
\end{tabular}
\end{table}

When regime persistence is relatively low ($p = 0.90$), switching occurs frequently. In this environment, MS-AR achieves the lowest RMSE (1.1426), while Global SARIMA (1.1695) and Rolling SARIMA (1.1875) perform worse. The differences across models are driven mainly by forecast error variance. MS-AR produces the lowest variance (1.3049), compared to 1.3676 and 1.4100 for the global and rolling specifications, respectively. Bias remains small for all methods and does not materially differentiate performance. Under frequent switching, explicit regime modeling primarily improves control of forecast dispersion.

At moderate persistence ($p = 0.95$), overall forecast errors decline relative to the $p = 0.90$ case. MS-AR continues to deliver the lowest RMSE (1.0778), though the gap relative to the rolling approach narrows. Forecast error variance decreases for all models, consistent with longer regime durations reducing instability. Bias fluctuates slightly but remains economically small. As regimes become more persistent, recent observations carry greater informational content about current dynamics, improving the relative performance of rolling estimation.

When persistence is high ($p = 0.99$), forecast errors decline further for all methods. MS-AR again achieves the lowest RMSE (1.0318), but the difference relative to Rolling SARIMA (1.0668) is smaller than under lower persistence. Forecast error variances converge across models, and bias becomes slightly negative for the global and rolling specifications, although magnitudes remain modest. With highly persistent regimes, rolling estimation approximates within-regime dynamics more effectively, reducing the advantage of explicit regime probability weighting.

Across all persistence levels, Global SARIMA consistently exhibits higher forecast error variance than the alternative methods. Differences in bias remain limited, indicating that forecast improvements stem primarily from reductions in dispersion rather than systematic mean correction. Overall, the recurring-break results show that explicit regime modeling yields consistent gains, particularly when switching is frequent, while rolling estimation becomes increasingly competitive as regimes grow more persistent.
