\section{Data Generating Process}

This section describes the data-generating processes (DGPs) used in the Monte Carlo experiments. All simulations are based on univariate AR(1) processes of length $T = 400$. In single-break designs, the structural break occurs at $T_b = 200$. Each experiment isolates a single dimension of structural instability—mean, variance, or autoregressive parameter—while holding all remaining features constant. All regimes satisfy $|\phi| < 1$, ensuring covariance stationarity.

Let $\{y_t\}_{t=1}^T$ denote the simulated process. The baseline specification is given by
\begin{equation}
y_t = \mu_t + \phi_t (y_{t-1} - \mu_t) + \varepsilon_t,
\end{equation}
where $\mu_t$ denotes the regime-dependent mean, $\phi_t$ the autoregressive coefficient, and $\varepsilon_t$ the innovation term.

Innovations are drawn either from a Gaussian distribution or from a standardized Student-$t$ distribution with degrees of freedom $\nu \in \{3,5\}$. In the latter case, shocks are rescaled to have unit variance to ensure comparability across innovation types. Unless stated otherwise, the innovation variance equals one.

\subsection{Single Structural Break Designs}

In the deterministic single-break setting, parameters shift once at $t = T_b$.

\subsubsection{Mean Break}

The mean break design is defined as
\begin{equation}
y_t = \mu_t + \phi (y_{t-1} - \mu_t) + \varepsilon_t,
\end{equation}
with constant persistence $\phi = 0.6$ and
\begin{equation}
\mu_t =
\begin{cases}
0, & t \le T_b, \\
2, & t > T_b.
\end{cases}
\end{equation}
The break therefore induces a discrete upward shift in the unconditional mean, while persistence and innovation variance remain unchanged.

\subsubsection{Variance Break}

Variance instability is introduced through
\begin{equation}
y_t = \mu + \phi (y_{t-1} - \mu) + \varepsilon_t,
\end{equation}
where $\mu = 0$, $\phi = 0.6$, and
\begin{equation}
\varepsilon_t \sim
\begin{cases}
\mathcal{D}(0, \sigma_1^2), & t \le T_b, \\
\mathcal{D}(0, \sigma_2^2), & t > T_b,
\end{cases}
\end{equation}
with $\sigma_1 = 1$ and $\sigma_2 = 2$. The structural break thus increases the innovation variance from $1$ to $4$ while leaving the conditional mean dynamics unchanged.

\subsubsection{Parameter Break}

In the parameter break design, persistence changes at $T_b$:
\begin{equation}
y_t = \phi_t y_{t-1} + \varepsilon_t,
\end{equation}
with
\begin{equation}
\phi_t =
\begin{cases}
0.2, & t \le T_b, \\
0.9, & t > T_b.
\end{cases}
\end{equation}
The break represents a transition from weak persistence to highly persistent near-unit-root dynamics, while the mean and innovation variance remain constant.

\subsection{Recurring (Markov-Switching) Break Designs}

To model recurring structural instability, a latent regime indicator $S_t \in \{0,1\}$ evolves according to a first-order Markov chain with transition probabilities
\begin{equation}
P(S_t = i \mid S_{t-1} = i) = p_{ii}.
\end{equation}

Unless otherwise specified, transition probabilities are symmetric, $p_{00} = p_{11} = 0.95$, implying an expected regime duration of approximately $1/(1-0.95) = 20$ periods.

\subsubsection{Recurring Mean Break}

The observation equation becomes
\begin{equation}
y_t = \mu_{S_t} + \phi (y_{t-1} - \mu_{S_t}) + \varepsilon_t,
\end{equation}
with $\mu_0 = 0$, $\mu_1 = 2$, and $\phi = 0.6$. Regime changes therefore induce stochastic shifts in the unconditional mean while persistence and variance remain constant.

\subsubsection{Recurring Variance Break}

Variance switching is modeled as
\begin{equation}
y_t = \mu + \phi (y_{t-1} - \mu) + \varepsilon_t,
\end{equation}
where $\mu = 0$, $\phi = 0.6$, and
\begin{equation}
\varepsilon_t \sim \mathcal{N}(0, \sigma_{S_t}^2),
\end{equation}
with $\sigma_1 = 1$ and $\sigma_2 = 2$. Regime transitions generate recurrent volatility shifts without altering the conditional mean dynamics.

\subsubsection{Recurring Parameter Break}

Persistence switching is defined by
\begin{equation}
y_t = \phi_{S_t} y_{t-1} + \varepsilon_t,
\end{equation}
where $\phi_0 = 0.2$ and $\phi_1 = 0.9$. 

In this case, regime persistence varies across experiments according to
\begin{equation}
p_{00} = p_{11} \in \{0.90, 0.95, 0.99\}.
\end{equation}
Higher persistence values imply longer regime durations and therefore stronger dynamic instability.

\subsection{Design Considerations}

Each DGP isolates one structural dimension—mean, variance, or persistence—while holding the remaining components fixed. Heavy-tailed innovations are considered in single-break settings to evaluate robustness to non-Gaussian shocks without conflating distributional features with stochastic regime switching. Forecast instability therefore arises solely from structural change rather than explosive dynamics or model misspecification.
