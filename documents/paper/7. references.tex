\begin{thebibliography}{99}

\bibitem{BaiPerron1998}
Bai, J., \& Perron, P. (1998).
Estimating and testing linear models with multiple structural changes.
\textit{Econometrica}, 66(1), 47--78.

\bibitem{BaiPerron2003}
Bai, J., \& Perron, P. (2003).
Computation and analysis of multiple structural change models.
\textit{Journal of Applied Econometrics}, 18(1), 1--22.

\bibitem{ClarkMcCracken2005}
Clark, T. E., \& McCracken, M. W. (2005).
The power of tests of predictive ability in the presence of structural breaks.
\textit{Journal of Econometrics}, 124(1), 1--31.

\bibitem{ClarkMcCracken2001}
Clark, T. E., \& McCracken, M. W. (2001).
Tests of equal forecast accuracy and encompassing for nested models.
\textit{Journal of Econometrics}, 105(1), 85--110.

\bibitem{ClementsHendry1998}
Clements, M. P., \& Hendry, D. F. (1998).
\textit{Forecasting Economic Time Series}.
Cambridge University Press.

\bibitem{ClementsHendry2006}
Clements, M. P., \& Hendry, D. F. (2006).
Forecasting with breaks.
In G. Elliott, C. W. J. Granger, \& A. Timmermann (Eds.),
\textit{Handbook of Economic Forecasting}, Vol. 1, 605--657.
Elsevier.

\bibitem{DieboldMariano1995}
Diebold, F. X., \& Mariano, R. S. (1995).
Comparing predictive accuracy.
\textit{Journal of Business \& Economic Statistics}, 13(3), 253--263.

\bibitem{Gardner1985}
Gardner, E. S. (1985).
Exponential smoothing: The state of the art.
\textit{Journal of Forecasting}, 4(1), 1--28.

\bibitem{Hamilton1989}
Hamilton, J. D. (1989).
A new approach to the economic analysis of nonstationary time series and the business cycle.
\textit{Econometrica}, 57(2), 357--384.

\bibitem{Hanninen2018}
H\"anninen, S. (2018).
Forecasting under structural breaks: Direct versus iterated forecasts.
\textit{Journal of Forecasting}, 37(5), 561--578.

\bibitem{Hansen2001}
Hansen, B. E. (2001).
The new econometrics of structural change: Dating breaks in U.S. labor productivity.
\textit{Journal of Economic Perspectives}, 15(4), 117--128.

\bibitem{Holt1957}
Holt, C. C. (2004).
Forecasting seasonals and trends by exponentially weighted moving averages.
\textit{International Journal of Forecasting}, 20(1), 5--10.
(Original work published 1957.)

\bibitem{HyndmanAthanasopoulos2018}
Hyndman, R. J., \& Athanasopoulos, G. (2018).
\textit{Forecasting: Principles and Practice} (2nd ed.).
OTexts.

\bibitem{InoueKilian2004}
Inoue, A., \& Kilian, L. (2004).
In-sample or out-of-sample tests of predictability: Which one should we use?
\textit{Econometric Reviews}, 23(4), 371--402.

\bibitem{PesaranTimmermann2007}
Pesaran, M. H., \& Timmermann, A. (2007).
Selection of estimation window in the presence of breaks.
\textit{Journal of Econometrics}, 137(1), 134--161.

\bibitem{Perron1989}
Perron, P. (1989).
The great crash, the oil price shock, and the unit root hypothesis.
\textit{Econometrica}, 57(6), 1361--1401.

\bibitem{PesaranPickTimmermann2011}
Pesaran, M. H., Pick, A., \& Timmermann, A. (2011).
Optimal forecasts in the presence of structural breaks.
\textit{Journal of Econometrics}, 164(1), 188--205.

\bibitem{PesaranTimmermann2013}
Pesaran, M. H., Pick, A., \& Timmermann, A. (2013).
Forecasting under structural breaks.
\textit{Journal of Econometrics}, 172(1), 1--2.

\bibitem{Rossi2013}
Rossi, B. (2013).
Advances in forecasting under instability.
In G. Elliott \& A. Timmermann (Eds.),
\textit{Handbook of Economic Forecasting}, Vol. 2, 1203--1324.
Elsevier.

\bibitem{StockWatson1996}
Stock, J. H., \& Watson, M. W. (1996).
Evidence on structural instability in macroeconomic time series relations.
\textit{Journal of Business \& Economic Statistics}, 14(1), 11--30.

\bibitem{StockWatson2003}
Stock, J. H., \& Watson, M. W. (2003).
Forecasting output and inflation: The role of asset prices.
\textit{Journal of Economic Literature}, 41(3), 788--829.

\bibitem{Tian2011}
Tian, Y. (2011).
Forecast combinations under structural breaks.
\textit{Journal of Forecasting}, 30(6), 625--648.

\bibitem{West1996}
West, K. D. (1996).
Asymptotic inference about predictive ability.
\textit{Econometrica}, 64(5), 1067--1084.

\bibitem{Winters1960}
Winters, P. R. (1960).
Forecasting sales by exponentially weighted moving averages.
\textit{Management Science}, 6(3), 324--342.

\end{thebibliography}
