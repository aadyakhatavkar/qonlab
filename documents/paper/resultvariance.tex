\subsection{5.3 Variance Breaks}

This subsection evaluates forecast performance when structural instability affects the innovation variance while the conditional mean dynamics remain unchanged. Results are reported for both deterministic single variance shifts and recurring variance regimes. As in the previous sections, emphasis is placed on RMSE, bias, forecast error variance, and density-based measures.

\subsubsection{5.3.1 Single Variance Break}

Under Gaussian innovations, differences across models are relatively small in terms of RMSE. The SARIMA averaged-window approach achieves the lowest RMSE (2.0518), closely followed by GARCH (2.0535). Global and Rolling SARIMA perform slightly worse. The ranking is similar for MAE. Bias remains modest across specifications, with values ranging between 0.1884 and 0.2094, indicating limited systematic forecast distortion. The primary differences arise in forecast error variance and log score. The averaged-window SARIMA exhibits the lowest error variance (4.1746), while Rolling SARIMA shows the highest (4.2430). Log scores are also slightly more favorable for the averaged-window model. Overall, under Gaussian variance shifts, no model dominates strongly, but approaches that allow partial adaptation to changing volatility perform marginally better.

When innovations follow a Student-$t$ distribution with three degrees of freedom, GARCH achieves the lowest RMSE (2.0311), although the difference relative to Global SARIMA is minimal. Forecast error variance is also lowest under GARCH (4.0423). In contrast, Rolling and averaged-window SARIMA show higher dispersion. Bias values are somewhat larger than in the Gaussian case, reflecting the influence of heavy-tailed shocks. The log predictive score is most favorable for GARCH, suggesting improved density calibration under volatility instability combined with heavy-tailed innovations. In this setting, explicitly modeling conditional heteroskedasticity provides measurable gains.

For Student-$t$ innovations with five degrees of freedom, overall forecast errors increase relative to the Gaussian case. The averaged-window SARIMA achieves the lowest RMSE (2.2381), though differences across models remain small. Error variances are higher than in the Gaussian design, and dispersion differences become more pronounced. GARCH no longer provides a clear advantage in RMSE, although its log score remains competitive. Rolling SARIMA continues to exhibit comparatively higher error variance. Across distributions, improvements in this single-break variance design are modest and primarily driven by differences in error dispersion rather than bias reduction.

Taken together, the single variance break results indicate that modeling conditional heteroskedasticity becomes more relevant as innovation distributions deviate from normality. However, when the variance shift is deterministic and occurs only once, the relative performance differences across models remain limited.

\subsubsection{5.3.2 Recurring Variance Break}

The recurring variance design introduces stochastic switching between low- and high-volatility regimes. In this environment, differences across models become more pronounced.

Global SARIMA achieves the lowest RMSE (1.6018), closely followed by the Markov-switching AR(1) model (1.6031). Rolling and averaged-window SARIMA perform slightly worse. Bias values are small and negative across all specifications, indicating mild underprediction but no substantial systematic distortion. The key distinction lies in forecast error variance. Global SARIMA produces the lowest dispersion (2.5532), whereas rolling and averaged-window approaches exhibit higher variance.

The log predictive score reveals an important pattern. Although Global SARIMA performs well in RMSE terms, its log score is less favorable than that of the Markov-switching model. The MS-AR(1) specification delivers the most favorable log score (-2.1097), indicating better calibration of predictive density under recurring volatility shifts. This suggests that explicitly modeling regime-dependent variance improves density forecasting even when point forecast gains are small.

Compared to the single-break case, forecast error variance is substantially lower in absolute terms, reflecting the stochastic rather than abrupt nature of regime changes. However, differences across models are more closely tied to density accuracy than to point forecast measures.

In the recurring variance environment, explicit regime modeling improves variance calibration, while point forecast differences remain moderate. Models that assume constant variance remain competitive in RMSE terms but may underestimate volatility dynamics, as reflected in less favorable log scores.
