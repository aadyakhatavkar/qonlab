\section{Forecasting Methods}

Forecasts are generated in a recursive one-step-ahead framework. At each forecast origin $t$, models are estimated using the available training sample and used to produce the forecast $\hat{y}_{t+1|t}$. This procedure is repeated throughout the out-of-sample period to ensure a consistent evaluation across break types.

Unless otherwise stated, the following core forecasting models are applied across all structural break environments.

\subsection{1. Core Forecasting Models}

\subsubsection*{(i) Global SARIMA}

A SARIMA$(1,0,1)(1,0,0)_{12}$ specification is estimated on the full training sample. The seasonal period $s = 12$ is imposed consistently across designs to allow for potential cyclical dynamics and to maintain comparability across break environments.

In lag-operator notation,
\begin{equation}
\Phi(L^{12}) \, \phi(L) y_t = \Theta(L) \varepsilon_t.
\end{equation}

Parameters are assumed constant over time and estimated using all available observations at each forecast origin. Because it pools information across regimes, this specification serves as a benchmark model under structural change.

\subsubsection*{(ii) Rolling SARIMA}

To allow for parameter adaptation, the same SARIMA$(1,0,1)(1,0,0)_{12}$ structure is estimated using a rolling window of fixed length $W$. In the simulations, the rolling window length is chosen to balance adaptability and estimation stability (e.g., $W = 80$ or $W = 100$, depending on the break design).

Formally,
\begin{equation}
\hat{y}_{t+1|t}^{(W)} = 
E\left( y_{t+1} \mid y_{t-W+1}, \dots, y_t \right).
\end{equation}

By restricting estimation to recent observations, the rolling approach reduces contamination from outdated regimes and improves responsiveness to structural shifts. However, this comes at the cost of higher estimation variance due to the smaller effective sample.

\subsection{2. Break-Specific Extensions}

Additional models are introduced depending on the form of instability.

\subsubsection*{A. Mean Break Designs}

\paragraph{(i) Markov-Switching Mean Model}

The conditional mean varies across regimes:
\begin{equation}
y_t = \mu_{S_t} + \phi (y_{t-1} - \mu_{S_t}) + \varepsilon_t.
\end{equation}

The one-step-ahead forecast is computed as a probability-weighted average across regimes:
\begin{equation}
\hat{y}_{t+1|t}
=
\sum_{i=0}^{1}
\pi_{t|t}(i)
\left[
\mu_i + \phi (y_t - \mu_i)
\right],
\end{equation}
where $\pi_{t|t}(i)$ denotes the filtered probability of regime $i$.

This model is designed to capture recurring shifts in the intercept while maintaining a common autoregressive structure.

\paragraph{(ii) Break Dummy (Oracle Specification)}

An exogenous dummy variable is included:
\begin{equation}
D_t =
\begin{cases}
0, & t \le T_b, \\
1, & t > T_b.
\end{cases}
\end{equation}

The dummy shifts the intercept after the known break date. Because the break timing is assumed known, this specification provides an upper performance bound rather than a feasible forecasting strategy.

\paragraph{(iii) Simple Exponential Smoothing (SES)}

Forecasts are generated recursively as
\begin{equation}
\hat{y}_{t+1|t}
=
\lambda y_t
+
(1-\lambda)\hat{y}_{t|t-1}.
\end{equation}

SES assigns geometrically declining weights to past observations and is particularly suited to level shifts. It provides a fully adaptive alternative to parametric models.

\subsubsection*{B. Parameter Break Designs}

To capture persistence instability, a Markov-switching AR(1) model is estimated:
\begin{equation}
y_t = \phi_{S_t} y_{t-1} + \varepsilon_t.
\end{equation}

The one-step-ahead forecast is constructed as
\begin{equation}
\hat{y}_{t+1|t}
=
\left(
\pi_{t|t}(0)\phi_0
+
\pi_{t|t}(1)\phi_1
\right)
y_t.
\end{equation}

This specification directly aligns with the recurring parameter-break DGP, where persistence shifts between low and high autoregressive regimes.

\subsubsection*{C. Variance Break Designs}

For volatility instability, models that produce both mean and variance forecasts are considered.

\paragraph{(i) GARCH(1,1)}

Conditional variance evolves according to
\begin{equation}
\sigma_t^2
=
\omega
+
\alpha_1 \varepsilon_{t-1}^2
+
\beta_1 \sigma_{t-1}^2.
\end{equation}

The one-step-ahead variance forecast is
\begin{equation}
\hat{\sigma}_{t+1|t}^2
=
\omega
+
\alpha_1 \varepsilon_t^2
+
\beta_1 \sigma_t^2.
\end{equation}

This model explicitly captures time-varying volatility and is therefore structurally consistent with variance-break environments.

\paragraph{(ii) Markov-Switching Variance Model}

Variance depends on the latent regime:
\begin{equation}
\varepsilon_t \sim \mathcal{N}(0,\sigma_{S_t}^2).
\end{equation}

The forecast variance is given by
\begin{equation}
\hat{\sigma}_{t+1|t}^2
=
\pi_{t|t}(0)\sigma_0^2
+
\pi_{t|t}(1)\sigma_1^2.
\end{equation}

This specification allows for recurring volatility regimes and provides a structural alternative to GARCH-based modeling.
